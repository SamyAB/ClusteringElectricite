\documentclass[a4paper,titlepage,12pt]{report}

\usepackage[utf8]{inputenc}
\usepackage{titlesec}%pour ne pas afficher "Chapitre X" en début de chapitre
\titleformat{\chapter}[display]
	  {\normalfont\bfseries}{}{0pt}{\Huge}
\usepackage[T1]{fontenc}
\usepackage[margin=1.2in]{geometry}
\usepackage[french]{babel}
\usepackage{verbatim}
\usepackage{graphicx}
\usepackage[hidelinks]{hyperref}
\usepackage{lmodern}
\setcounter{secnumdepth}{3}
\usepackage{enumitem}
\usepackage{wasysym}
\usepackage{pifont}
\usepackage[french,onelanguage]{algorithm2e} %for psuedo code
\usepackage[labelfont=bf]{caption}
\captionsetup[figure]{labelfont=bf} %pour mettre les Figure X.X en gras


\title{Rapport}
\author{AIT BACHIR Samy}

\pagestyle{headings}

\begin{document}
\chapter{Introduction}
	Dans ce projet il est demandé d'effectuer une classification sur les données de consommation d'électricité de 2914 logements irlandais, et ce afin d'obtenir des classes homogènes, séparant les logements en plusieurs profils, selon leurs consommation.\newline
	
	Les différents algorithmes utilisés, vus durant le cours d'apprentissage non supervisé, sont :
	\begin{itemize}
		\item La classification hiérarchique ascendante(CAH).
		\item L'algorithme des centres mobiles (K-means).
		\item L'algorithme Self Organising Map (SOM).
	\end{itemize}
	
	Les algorithmes sont directement utilisés sur les données fournies, car celles-ci ont déjà été normalisées.\newline
	
	Dans les prochains chapitres, ce rapport décrira l'utilisation des différents algorithmes sous R sur le jeu de données, puis discutera les résultats, premièrement sur l'ensemble des données, puis dans un second temps uniquement sur les données représentant les heures des jours ouvrés. Ce document se termine par une brève conclusion. 
	 
\chapter{Partitionnement et analyse sur l'ensemble des données}
	\section{Algorithmes de classification}
		\subsection{Classification Hiérarchique Ascendante CAH}
			Les algorithme de classification hiérarchique ascendante sont des algorithmes qui ormes des clusters à chaque itération en combinant plusieurs clusters, se basant sur la notion de distance entre ceux-ci.
			
			À l'initialisation de l'algorithme, chaque instance représente un cluster. à la fin de l'algorithme, toutes les instances se trouve dans un seul et même cluster.
			
			Ici, la distance utilisée est la distance euclidienne. Pour commencer les distances sont calculées deux à deux entre instances et sont sauvegardées dans une matrice symétrique dite "Matrice de distance".
			
			La méthode de combinaison des clusters utilisée ici est appelée méthode de Ward, cette méthode permet, à chaque étape, d’augmenter un minimum l'inertie intra-classe, et ce jusqu'à ce que celle-ci atteigne la valeur de l'inertie totale.
			
		\subsection{Algorithme des Centres Mobiles K-means}
			L'algorithme K-means ne donne pas à priori de moyen d'estimer le nombre de classes, par conséquent, dans ce projet il a été pari pris de tester un nombre variables de clusters, mais en se basant sur le nombre de clusters donnés par la CAH, et donc de faire varier le K entre 2 et 5.
			
			L'algorithme ayant une initialisation aléatoire, les algorithmes sont relancés plusieurs fois. 
		
		\subsection{Algorithme Self Organising Map}
			Pour l'algorithme MAP, plusieurs types de grille ont été testées (une dimension, deux dimensions), puis la grille finalement retenue est une grille à deux dimensions, de tailles 10x10.
									
	\section{Représentations et analyses des résultats}
		\subsection{CAH}
			Le dendrogramme retourné par la méthode nous indique que le nombre de classe serait 2, par conséquent, on effectue une coupure sur la dernière étape de ce graphique, pour obtenir les deux classes dont les centres sont montrés sur la figure.
		
			D'après les moyennes des clusters obtenues, on peut séparer les foyers en deux parties distinctes, une partie de foyer qui a une forte consommation, et une autre une basse consommation. De plus on remarque que les foyers du premier clusters (en noir), baissent leurs consommations en milieu de journée, et que celle ci remonte au fil de l'après midi pour atteindre son pique en début de soirée.
		
			Les foyers à forte consommation (en rouge) voient leurs consommations augmenter, mais de façon continue, ceci dit cette augmentation stagne quelques en milieu de journée.
		
		\subsection{K-means}
			Pour chacun des K utilisé un graphe a été généré par souci de concision du rapport seul le plot jugé le plus pertinent a été ajouté à celui-ci.
			
			%KM4 PNG
			
			Nous voyons bien des classes se rapportant a celles détectées par la CAH précédemment, mais on arrive à voir différents profils intermédiaires.
			
		\subsection{SOM}
			La map obtenue par l'algorithme n'est pas forcément utilisable telle qu'elle, un K-means est exécuté par dessus, avec 3 classes.
		
\chapter{Partitionnement et analyse sur les données hors week-end}
	\section{Algorithmes de classification}
		Les mêmes algor
	\section{Représentations et analyses des résultats}
\chapter{Conclusion}
\end{document}